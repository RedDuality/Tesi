\section*{Introduzione}
\addcontentsline{toc}{section}{Introduzione}
L'informatica svolge da anni ruoli sempre più essenziali nelle gestioni aziendali ma anche nella vita di tutti i giorni. 
Per applicazioni con particolari requisiti di sicurezza, migliaia di utenti o specifiche garanzie di servizio, la sicurezza, 
la scalabilità e l'affidabiliità non sono opzioni ma esigenze. 
Tali funzionalità possono essere ottenute tramite infrastrutture progettate e costruite autonomamente, 
ma richiedono l'investimento di risorse importanti, basti pensare alla progettazione, al deployment su macchine fisiche e alla relativa manutenzione, sia fisica che logica.
Per fare un esempio, per poter garantire l'utilizzo anche in fase di richieste elevate, è presupporto che si mantenga un'infrastruttura che statisticamente verrà usata in minima parte.
La maggior parte dei sistemi richiede operazioni che si discostano dalle caratteristiche centrali dell'applicazione, 
quali il monitoraggio degli eventi o tutta la gestione della sicurezza.\\
\\
I cloud providers nascono con la finalità di proporre piattaforme che risolvano la gran parte dei problemi comuni alle diverse applicazioni,
nascondendone e astraendone la complessità.
Questo comporta vantaggi per sia per lo sviluppatore che per il cloud provider. 
Lo sviluppatore può concentrarsi sulla logica applicativa, scegliendo il prodotto/i che più si addice alle sue esigenze, 
preoccupandosi solamente di fare in modo che la configurazione dei servizi sia corretta. 
Non deve più preoccuparsi per la gestione fisica delle macchine, e può, nella maggior parte dei casi, pagare solo per le risorse che utilizza.
Il cloud provider, vendendo lo stesso prodotto a più clienti, concentra le risorse richieste per la manutenzione del servizio
e ammorta i volumi computazionali eventualmente richiesti per la gestione di carichi elevati, 
guadagnando sulle risorse risparmiate rispetto al caso in cui ogni servizio fosse stato gestito autonomamente.
I cloud providers forniscono molteplici servizi con capacità e responsabilità diverse, specifiche per varie esigenze, 
avvicinandosi il più possibile ai bisogni specifici dei clienti.\\
\\
La facilità di configurazione e il costo ridotto iniziale dei servizi proposti rende possibile anche a realtà di piccole e medie dimensioni 
creare progetti con capacità, ambizioni e qualità superiori a quelli che le loro normali risorse permetterebbero.
Per questa ragione, anche in fase di progetto e con i prototipi iniziali, conviene basarsi su risorse in cloud, 
integrando da subito funzionalità comunque eventualmente necessarie e 
identificando il più precocemente possibile gli strumenti più adatti all'applicazione che si sta costruendo.\\
\\
Nella scelta dei servizi offerti dai cloud providers, si rivela però facile confondersi tra le tante opportunità proposte
che appariono in un primo giudizio molto simili ma, magari nate per finalità particolarmente differenti,
potrebbero risulatere implementate con architetture molto diverse, determinandone potenzialità e limiti.
Risulta quindi fondamentale saper individuare il servizio che più si addice alle proprie necessità, 
distinguendolo tra gli altri per le differenze essenziali che comporteranno un vantaggio nell'esecuzione del progetto.\\
\\
Tra i rischi maggiori che si corrono implementando un'applicazione tramite infrastrutture in cloud,
oltre al perdere il controllo del budget dati i costi variabili, sussiste la scelta sbagliata dei servizi da sfruttare, che, 
magari illundendo inizialmente un corretto funzionamento, può far sorgere problemi di integrazione o di funzionalità 
più avanti nella vita del prodotto.
Per quanto si presentino come soluzioni indipendenti e virtualizzate, 
la scelta sbagliata di un componente può comportare la riscrittura di parti intere del programma,
dal momento che ogni risorsa richiede un approccio differente.
La scelta corretta di un componente può avvenire solo avendo ben chiare le necessità architetturali e le particolarità del prodotto che si vuole implementare.\\
\\
Oltre a trovare lo strumento offerto più inerente alle proprie esigenze, 
il codice implementato su tali strumenti deve rispondere alle potenzialità e alle capacità che è in grado di offrire, 
adattando le tecnologie alla soluzione ricercata.
Una struttura sicura, scalabile ed affidabile, infatti, lo è tanto grazie alle tecnologie da usare quanto alle scelte ingegneristiche di come usarle.
L'individuazione di suddette necessità e particolarità del prodotto deve perciò avvenire congiuntamente all'analisi richiesta per sviluppare il codice.\\
\\
In questa tesi si analizzeranno le scelte prese durante lo sviluppo di un'applicazione con requisiti di alta scalabilità facendo particolare attenzione ai  \\

\clearpage
\phantomsection
\subsection*{Descrizione dei capitoli}
\addcontentsline{toc}{subsection}{Descrizione dei capitoli}

Nell'ingegneria del software, la branca che si occupa di sviluppare un prodotto partendo da un'idea iniziale, 
si individuano diverse fasi per la creazione di un software resistente e mantenibile, oltre che efficace.\\

    La prima fase consiste con l'abstract, in cui si sintetizza l'idea generale del progetto, 
    specificando le funzionalità principali e la visione d'insieme dell'applicazione.
    Segue il documento dei requisiti, che analizza l'abstract e ne estrae in maniera formale i requisiti e introducendo i casi d'uso, 
    ovvero tutte le azioni che il programma può compiere.\\
    L'analisi del problema deduce una struttura iniziale e inizia a definire il comportamento generale del programma.\\
    Il documento dei requisiti e l'analisi vengono elaborati in collaborazione con il committente, 
    per assicurarsi che le richieste siano uniformemente intese da entrambe le parti.
    In fase di progettazione vengono prese decisioni ad alto livello indipendenti dalle tecnologie specifiche da utilizzare, 
    identificandone però le caratteristiche necessarie. 
    In questa fase si individuano i possibili punti critici e le particolarità richieste al sistema.
    Si definiscono quindi il tipo di architettura, la struttura del sistema e la sua interazione tra le parti.
    La fase di implementazione documenta le scelte applicate sia a livello tecnologico che a livello di realizzazione. 
    Seguendo le scelte prese in fase di progettazione, dettaglia le scelte architetturali, delle diverse partei del codice e della loro interzione.
Seguendo lo stesso schema, per lo sviluppo di un sistema che presenti nei requisiti l'essere scalabile ed affidabile, 
le scelte relative necessarie emergono, vendono analizzate e applicate in linea con le altre scelte progettuali del programma.\\
\clearpage