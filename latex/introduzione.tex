

\section*{Introduzione}
\addcontentsline{toc}{section}{Introduzione}


In un contesto sempre più connesso, la crescente quantità di contatti, la rapidità delle comunicazioni e l'accesso universale alle informazioni rendono la creazione, l'organizzazione e la partecipazione ad eventi estremamente facile, ma al contempo generano un ambiente frenetico e spesso dispersivo.\\
Risulta infatti difficile seguire tutte le opportunità a cui si potrebbe partecipare, considerando le numerose occasioni che si presentano quotidianamente, rischiando di ricordarsene solo successivamente, una volta passate. Basti pensare, ad esempio, alle riunioni di lavoro, alle serate con amici, agli appuntamenti informali per un caffè, ma anche a eventi più strutturati come fiere, convention aziendali, concerti, partite sportive o mostre di artisti che visitano occasionalmente la città.\\
Questi eventi possono sovrapporsi, causando dimenticanze o conflitti di pianificazione, con il rischio di delusione o frustrazione. Quando si è invitati a un evento, può capitare di essere già impegnati, o di trovarsi in attesa di una conferma da parte di altri contatti. In questi casi, la gestione degli impegni diventa complessa: spesso si conferma la partecipazione senza considerare possibili sovrapposizioni, o dimenticandosi, per poi dover scegliere e disdire all'ultimo momento. \\
D’altra parte, anche quando si desidera proporre un evento, la ricerca di un'attività interessante può diventare un compito arduo, con la necessità di consultare numerosi profili social di locali e attività, senza avere inoltre la certezza che gli altri siano disponibili.\\
Tali problemi si acuiscono ulteriormente quando si tratta di organizzare eventi di gruppo, dove bisogna allineare gli impegni di più persone. \\
In questo contesto, emergono la necessità e l'opportunità di sviluppare uno strumento che semplifichi la proposta e la gestione degli eventi, separando il momento della proposta da quello della conferma di partecipazione. In tal modo, gli utenti possono valutare la disponibilità degli altri prima di impegnarsi definitivamente, facilitando l'invito e la partecipazione.\\ 
\clearpage
In risposta a tali richieste è nata Wyd, un'applicazione che permette ai clienti di organizzare i propri impegni, siano essi confermati oppure proposti. Essa permette anche di rendere più intuitiva la ricerca di eventi attraverso la creazione di uno spazio virtuale centralizzato dove gli utenti possano pubblicare e consultare tutti gli eventi disponibili, diminuendo l’eventualità di perderne qualcuno. 
La funzionalità chiave di questo progetto si fonda sull'idea di affiancare alla tradizionale agenda degli impegni certi un calendario separato, che mostri tutti gli eventi a cui si potrebbe partecipare. Una volta confermata la partecipazione a un evento, questo verrà spostato automaticamente nell'agenda personale dell'utente.
Gli eventi creati potranno essere condivisi con persone o gruppi, permettendo di visualizzare le conferme di partecipazione. 
Inoltre, considerando l'importanza della condivisione di contenuti multimediali, questo progetto prevede la possibilità di condividere foto e video con tutti i partecipanti all'evento, attraverso la generazione di link per applicazioni esterne o grazie all'ausilio di gruppi di profili. Al termine dell’evento,
l'applicazione carica automaticamente le foto scattate durante l'evento,
per allegarle a seguito della conferma dell'utente.\\
\\
Come anticipato, la realizzazione di un progetto come Wyd implica la risoluzione e la gestione di diverse problematiche tecniche. 
In primo luogo, garantire la persistenza dell'agenda dell'utente, che deve essere aggiornata e mantenuta in modo affidabile e coerente, tale da permettere per un uso distribuito del servizio. 
Inoltre, la funzionalità di condivisione degli eventi richiede l'aggiornamento in tempo reale di tutti gli utenti coinvolti, al fine di rimanere sempre aggiornati. 
Infine, il caricamento e il salvataggio delle foto introduce la necessità di gestire richieste di archiviazione di dimensioni significative.
In questa prospettiva, risulta fondamentale esplorare il procedimento di analisi e progettazione che ha portato allo sviluppo di un applicativo multipiattaforma efficace, affidabile e scalabile, in grado di soddisfare le esigenze degli utenti nell’ambito dell'organizzazione di eventi condivisi e della condivisione multimediale automatica in tempo reale.
Seguendo l’esame dei requisiti necessari, l’analisi dello sviluppo del progetto affronterà in una prima parte le scelte infrastrutturali che hanno portato a definire la struttura centrale dell’applicazione, spiegando brevemente i requisiti e le soluzioni trovate. 
Successivamente si osserverà lo studio effettuato per gestire la memoria, in quanto fattore che più incide sulle prestazioni. Particolare attenzione è stata dedicata, infatti, a determinare le tecnologie e i metodi che meglio corrispondono alle esigenze derivate dal salvataggio e dall’interazione logica degli elementi.
Infine, per permettere la gestione delle immagini, che introducono problematiche impattanti sia sulle dimensioni delle richieste sia sull'integrazione con la persistenza, una terza parte si concentrerà sulle scelte implementative adottate per l’inserimento delle funzionalità.


\clearpage
