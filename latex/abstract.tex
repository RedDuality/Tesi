

\section*{Abstract}


Lo sviluppo di un applicativo multipiattaforma diretto all’organizzazione di eventi condivisi, caratterizzato in particolare dalla condivisione multimediale in tempo reale, 
richiede opportune capacità di scalabilità, atte a garantire una risposta efficace anche con alti volumi di richieste, offrendo prestazioni ottimali. 
Le tecnologie cloud, con la loro disponibilità pressoché illimitata di risorse e alla completa e continua garanzia  di manutenzione, 
offrono l'architettura ideale per il supporto di simili progetti.\\
\\
Tuttavia, l'integrazione tra la logica applicativa ed i molteplici servizi cloud, 
insieme alla gestione delle loro interazioni reciproche, comporta sfide specifiche, in particolare legate all'ottimizzazione di tutte le risorse.\\
L’individuazione e la selezione delle soluzioni tecnologiche più adatte per ogni obiettivo, 
e l'adozione delle migliori pratiche progettuali devono procedere parallelamente con lo sviluppo del codice, al fine di sfruttare efficacemente le potenzialità offerte.\\
\\
In tale prospettiva, questa tesi illustra le scelte progettuali e implementative adottate nello sviluppo dell'applicativo in questione, 
evidenziando l’impatto dell'integrazione delle risorse cloud sul risultato finale.\\


\clearpage