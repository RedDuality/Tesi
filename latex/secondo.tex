\section{ Persistenza}

L’utente deve essere in grado di consultare in qualunque momento il suo calendario, per visualizzare gli impegni più urgenti e pianificare il suo tempo. Deve dunque essere possibile trovare e mostrare nel minor tempo possibile i dati relativi alla sua agenda. Nel caso in cui la tecnologia lo permetta, il salvataggio degli impegni sul dispositivo favorisce i tempi di caricamento, ma non garantisce una persistenza dei dati a lungo termine che sia anche fruibile dagli altri dispositivi.

La gestione della memoria ha quindi il compito di definire una fonte principale ed autoritaria dei dati, alla quale tutti i dispositivi possano fare affidamento per recuperare le informazioni di cui hanno bisogno. Segue la necessità della realizzazione di un meccanismo per la sincronizzazione dei dati salvati in locale con la loro controparte ufficiale. 
La struttura deve essere creata tenendo conto del dominio su cui opera, per poter garantire efficienza e solidità nelle interazioni, ed essere quindi resistente ad eventuali volumi importanti di richieste concorrenti.

Bisogna inoltre considerare che, per la natura condivisa dell'applicazione, gli elementi possono subire modifiche anche da altri utenti, e si aggiunge di conseguenza la problematica di mantenere aggiornati non solo i dispositivi collegati ad un utente ma anche tutti gli altri dispositivi connessi degli utenti interessati.





L’accesso ai dati richiede un punto di riferimento chiaro ed affidabile, per fornire una fonte primaria affidabile delle informazioni. 
Il salvataggio di copie dei dati sulle memorie locali dei dispositivi permette invece una risposta veloce e disponibile agli utenti, che consente di nascondere i ritardi dovuti al caricamento delle informazioni.

TODO introduzione alla memoria locale

La presenza di una memoria unica principale garantisce l’autorevolezza di una fonte a cui fare riferimento per prendere i dati ufficiali. questo comporta il problema di mantenere aggiornati i dati locali(responsabilità del client) ma anche di distribuire le modifiche apportate ai device interessati(responsabilità del server).

\clearpage
\subsection{Memoria principale}

	

Nell’implementazione di applicazioni scalabili il salvataggio dei dati può avvenire su risorse distribuite o centralizzate. L’organizzazione della memoria in maniera distribuita, nonostante molteplici vantaggi quali la resistenza a danni di una singola fonte, che permette di evitare gli eventuali disservizi, la riduzione della memoria necessaria per garantire il servizio, e la propensione alla scalabilità, esige la creazione di un’infrastruttura complessa per garantire la persistenza, l’affidabilità e la consistenza delle informazioni. A meno di requisiti che rendano necessaria la distribuzione totale o parziale della memoria, si possono ottenere risultati prestazionali migliori con strutture più semplici utilizzando un database centrale. 

\subsubsection{ Database e scalabilità}

I database si suddividono in due grandi macro categorie: relazionali e non relazionali. I database relazionali presentano strutture rigide che rendono però possibile collegare le risorse tra loro in tempi molto ristretti. I database non relazionali, viceversa, permettono strutture anche molto disomogenee ma accoppiano debolmente le entità presenti. Quest’ultima capacità fornisce ai database non relazionali una maggiore scalabilità orizzontale, rendendoli generalmente più adatto ad alti volumi di richieste.

La scelta del database più appropriato ricade però sulle esigenze specifiche del progetto. 
I database relazionali sono gli unici che garantiscono proprietà di atomicità, consistenza, isolamento e durabilità(ACID), rese possibili grazie al meccanismo delle transazioni, che comportano un blocco temporaneo all’accesso della risorsa. 
Dal punto di vista della scalabilità, invece, gli aspetti principali da prendere in considerazione sono la proporzione tra le operazioni di lettura e di scrittura, la natura dei dati e la relazione tra le entità. 

Le operazioni di lettura possono essere soddisfatte efficacemente nonostante un alto volume di richieste, grazie anche alla possibilità di creazione di copie e alla parallelizzazione dei processi.
Le operazioni di scrittura causano invece ritardi in quanto comportano il blocco all’accesso della risorsa fino alla fine dell’aggiornamento. Questo può provocare(in base alla tipologia del database) l’invalidità di tutte le copie fino a quando, a seguito del termine della scrittura, l’aggiornamento non è stato propagato. 
Grazie all’accoppiamento debole delle risorse e a requisiti meno stringenti sulle proprietà ACID, i database non relazionali rispondono meglio in situazioni in cui le operazioni di scrittura sono frequenti. 

Se sussiste l’esigenza di strutture con dati opzionali o variabili i database non relazionali si rivelano più duttili ed efficaci. Permettono infatti di salvare i dati sotto forma di documenti, che possono variare le loro proprietà senza dover modificare la struttura del database. Tale necessità avviene generalmente però in casi particolari, che coprono parzialmente il dominio. 



Nell’analisi della relazione tra le entità occorre prendere in considerazione la sproporzione tra la quantità di richieste relativa a ciascun elemento, oltre al carico computazionale di ciascuna richiesta. 
L’operazione di unione è l’azione che più ostacola e rallenta il recupero dei dati, impattando significativamente sulle prestazioni delle richieste.
Durante le operazioni di unione vengono incrociati i dati di vari elementi per restituire un oggetto coerente che presenti tutte le proprietà necessarie, originariamente distribuite in molteplici tabelle. La ricerca e il recupero dei dati tra più tabelle sono azioni computazionalmente pesanti che influiscono sul tempo di risposta finale. Per questo motivo si cerca di ridurre il più possibile le richieste che comportano l’incrocio di dati da tabelle diverse. 

Le relazioni tra elementi possono essere di tipo uno a uno, uno a molti, e molti a molti. Nelle relazioni uno a uno il recupero dei dati è diretto e richiede uno sforzo computazionale limitato. Nei casi uno a molti e molti a molti è necessario valutare se sia possibile evitare o quantomeno ridurre l’impiego delle operazioni di join: una delle soluzioni possibili offerte dai database non relazionali è il salvataggio per copia del dato all’interno dell’oggetto. Questa operazione risulta efficiente se le richieste sono sproporzionate verso una delle due parti.

Ad esempio, in una relazione uno a molti, nel caso in cui la richiesta di quell’elemento non sia frequente ma sia importante ottenere gli elementi collegati; conviene copiare gli oggetti relativi all’interno dell’elemento singolo. L’impostazione inversa, in cui si copia il singolo all’interno dei molti elementi, comporterebbe l’ispezione di tutti i componenti esistenti alla ricerca di quelli che contengono l’elemento dato. 
Se però, viceversa, sono frequenti le richieste relative agli elementi multipli, e la loro relazione è importante, conviene copiare il singolo all'interno di detti elementi, per evitarne il recupero ogni volta.

I database relazionali non permettono la copia di elementi all’interno di altri, ma sono ottimizzati per unire tra loro tabelle, richiedendo comunque tempistiche e capacità elaborative non indifferenti. 

Tuttavia, non vi è alcun vincolo che impedisca l’affiancamento di database di tipologia diversa per rispondere a esigenze specifiche e sfruttare i punti di forza di entrambe le tecnologie.
	
\subsubsection{ Analisi del dominio}

La scelta del database deriva da un’attenta analisi delle principali interazioni tra gli elementi del dominio. Il dominio descrive i componenti dell’applicazione e le loro relazioni. In particolare, vengono espresse le dipendenze, i rapporti reciproci e le cardinalità delle relazioni. A titolo esemplificativo, dal diagramma del dominio si deduce che ad un oggetto Event possono corrispondere più oggetti Photo, ma soprattutto che l’oggetto Photo è strettamente legato all’oggetto Event.

\begin{figure}[h!]
    \centering
    \includegraphics[width=\textwidth]{ProgettoDominioServer.png}
    \caption{ Diagramma del dominio}
\end{figure}
Le richieste riguardo alle informazioni tra Account, User e Profile, con i relativi UserRole, vengono eseguite all’avvio del programma, per poi essere mantenute in memoria locale. Le modifiche a questi elementi sono sporadiche.
Allo stesso modo gli oggetti Group vengono recuperati solo all’avvio dell’applicazione, e, salvo rari aggiornamenti, non occupano ulteriormente lo spazio delle richieste.

La maggioranza delle richieste verterà sull’ottenimento dei dati relativi agli Event e ai Profile. Infatti ad ogni avvio dell’applicazione sarà richiesto di recuperare gli eventi di ogni profilo, mentre, ogni volta che si apre il dettaglio di un evento, sarà necessario recuperare i rispettivi dati, includendo i profili associati. Si prevede che la cardinalità dei profili associati ad un evento risieda nell’ordine delle decine, mentre agli eventi che si associano ai profili l’ordine di grandezza previsto risiede nelle migliaia. 

La relazione che accomuna Event e Profile è di tipologia molti a molti, identificata con l’oggetto ProfileEvent. L’elemento ProfileEvent, oltre a descrivere la relazione, contiene la proprietà Confirmed, che indica la conferma di partecipazione di un profilo ad un evento. Vista la natura del progetto, la proprietà Confirmed, assieme ai dati degli Event, sarà tra i campi che subiranno modifiche con maggiore frequenza.

Riassumendo, le operazioni principali che vertono sulle prestazioni del database sono il recupero degli eventi relativi ai profili, il recupero dei profili collegati agli eventi, l’aggiornamento dei campi degli oggetti Event e la modifica del campo Confirmed relativa ai ProfileEvent.

Nonostante sia più probabile che venga richiesto il dettaglio di un evento, e quindi sia più frequente il dover recuperare i Profile relativi all’Event, la proporzione delle richieste prevista non giustifica lo sbilanciamento della relazione sugli Event.
Infatti, se si salvassero tutti i ProfileEvent sull’oggetto Event, le richieste degli eventi appartenenti ai profili, sebbene meno frequenti, richiederebbero l’ispezione di tutti gli Event alla ricerca del Profile indicato. 
Infine, se si duplicasse l’oggetto ProfileEvent, oltre che nella sua tabella originaria, anche sugli Event la sua creazione, la sua eliminazione e la modifica del campo Confirmed richiederebbero il doppio delle scritture.

Vista quindi la necessità di letture frequenti da entrambe le parti di una relazione molti a molti e la necessità di scrittura dell’oggetto che la identifica, si individua in un database relazionale la soluzione più efficace per il dominio del progetto. Infatti, i database relazionali sono ottimizzati sulle operazioni di unione tra tabelle, garantendo velocità di recupero in lettura da entrambe le parti. Permettendo la separazione delle tabelle e gestendo gli oggetti ProfileEvent come entità indipendenti, la modifica del campo Confirmed comporta il blocco del solo elemento ProfileEvent. Le caratteristiche ACID forniscono inoltre uno stato centrale per tutta l’applicazione, che, per quanto non strettamente necessario, garantisce l’uniformità delle informazioni per tutti gli utenti.
\clearpage
\subsubsection{Scelta del database}

Azure offre un’ampia scelta di database relazionali che possono essere integrati con il resto dell’ecosistema. Oltre alla tecnologia proposta, nella scelta del database più adatto bisogna considerare soprattutto l’integrazione con servizi accessori, le particolarità del server su cui viene eseguito, e i costi che si andrà ad affrontare.

\begin{figure}[h!]
    \centering
    \includegraphics[width=\textwidth]{Possibilitadelcloud.png}
    \caption{Proposte di Azure per i database relazionali}
\end{figure}	


A meno di necessità particolari che richiedono l’utilizzo di una tecnologia implementata da uno specifico database, la grande sovrapposizione di funzionalità delle diverse offerte di database relazionali presenti sul mercato garantisce il  soddisfacimento delle necessità rilevate dall’analisi del dominio indipendentemente dalla tecnologia proposta dal servizio.

La grande differenza tra i vari servizi sta nelle proprietà del server incaricato di fornire il potere computazionale necessario per l’esecuzione. L’architettura del server e la sua integrazione con la tecnologia del database, infatti, determinano l’effettiva capacità di scalabilità del servizio. 

Si intende scalabilità verticale la capacità di aumentare le risorse della stessa macchina in cui si esegue il codice. La scalabilità verticale viene definita nel momento di creazione del servizio, in cui si determinano le risorse da dedicare alla macchina che esegue il programma. Trattandosi di macchine virtualizzate, è sempre possibile in un secondo momento aumentare le prestazioni in caso di necessità.

Per scalabilità orizzontale si intende invece la capacità di delegare il carico di lavoro ad altre macchine, eventualmente coordinando le modifiche. Questo permette una risposta alle richieste più resistente, riducendo il rischio di colli di bottiglia che potrebbero venirsi a formare nell’utilizzo di un nodo singolo.
La scalabilità orizzontale richiede però l’implementazione di tecnologie apposite integrate con il database che permettano l’esecuzione in nodi fisici differenti. 

Una volta individuata la tecnologia adatta e il livello di scalabilità desiderati, è bene considerare le altre necessità o le opportunità aggiuntive generate dalla presenza di un database nel progetto. 

L’alta disponibilità(HA) è la proprietà di garantire l’accesso al servizio nonostante i guasti. Ad esempio, si può mantenere una macchina identica al server principale in grado di replicare il servizio, spostando il carico in caso di guasto del server principale. Si misura in “numero di nove”, ovvero la quantità di nove presenti nella percentuale del tempo per il quale si garantisce la disponibilità del servizio.
I servizi offrono diverse qualità di HA, in base alle funzionalità desiderate.

Alcuni servizi possono presentare offerte di backup per riportare il server nello stesso stato di qualche momento precedente. Questo permette il ripristino del sistema ad un punto precedente rispetto all’avvenimento di eventuali errori o guasti del sistema.

Inoltre, Azure mette a disposizione molteplici servizi accessori che possono essere uniti al servizio. Questo permette di estendere le potenzialità del database tramite  l’analisi e il monitoraggio dei dati, generando prestazioni aggiuntive o integrando i dati per lo sviluppo di altre tecnologie.

Infine è necessario controllare i costi che le scelte progettuali e prestazionali hanno comportato: pur generalmente legati al consumo effettivo delle risorse, e quindi in grado di fornire solo una previsione del costo finale, ogni decisione presa conduce ad un possibile aumento di prezzo, ed è quindi bene controllare che le risorse selezionate siano effettivamente  necessarie a soddisfare i requisiti del progetto.

Riducendo al minimo i costi, visto l’utilizzo iniziale dell’applicazione, e nessuna necessità tecnologica specifica, la scelta del database per la persistenza del progetto è ricaduta su Azure SQL Database. 
Presentando un database relazionale di tecnologia proprietaria di Microsoft, Azure SQL database esegue su un solo nodo, fornendo però la possibilità di  scalare  verticalmente tramite la possibilità di modificare le risorse assegnate in ogni momento, garantendo comunque prestazioni soddisfacenti per il servizio.
Al server principale è stata affiancata una replica che rimane costantemente aggiornata. Situata in una località differente dal server principale, garantisce alta disponibilità continuando a fornire i servizi anche in caso di malfunzionamenti al server principale.
\subsubsection{ Limitazioni dei database relazionali}

La scelta di un database relazionale può però comportare limitazioni a livello prestazionale. Il più grande tallone d'Achille dei database relazionali è il numero limitato di connessioni contemporanee permesso. Questo comporta un numero massimo di richieste in contemporanea che il database può gestire, minacciando la scalabilità. 

Per ovviare ai problemi di scalabilità ci sono diverse soluzioni non esclusive che possono migliorare le prestazioni. Sicuramente si deve limitare al minimo il tempo in cui ogni richiesta mantiene la connessione, distinguendo nel codice momenti precisi e definiti in cui vengono richieste le modifiche al database.   

Inoltre, si può interporre un livello di caching tra la logica applicativa e le richieste al database. Il livello di caching si occupa di gestire le richieste al database fornendo e duplicando le risposte che possiede già in memoria, eventualmente centralizzando le richieste in caso i dati siano invece da recuperare. Per i dati in scrittura, invece, salva temporaneamente le modifiche richieste, aggiornando subito la memoria locale, per poi apportare le modifiche al database in momenti di carico ridotto. Garantisce così un tempo di risposta e di propagazione degli aggiornamenti ridotto. Questo consente di alleviare le richieste al database, estendendo di molto le prestazioni fornite.

Nel caso in cui però fossero necessarie ulteriori prestazioni, se il dominio e i requisiti lo permettono, si può eventualmente delegare ad un database non relazionale le modifiche ai dati e alle relazioni che non necessitano delle qualità ACID ma richiedono un’alta frequenza di scrittura. 

Infine, se fosse necessario mantenere un database relazionale, per aumentare le prestazioni conviene cambiare architettura e adottarne una che supporti la scalabilità orizzontale, per suddividere il carico su più nodi.




\subsubsection{ Integrazione con C\#}

La scelta di un database relazionale per la persistenza ha comportato sviluppi progettuali precisi. In primis si rende necessario tradurre il dominio in componenti relazionali che possano essere espressi e salvati nelle tabelle del database.

\begin{figure}[h!]
    \centering
    \includegraphics[width=\textwidth]{ProgettoDiagrammaER.png}
    \caption{Diagramma Entità - Relzione del dominio}
\end{figure}	

Si creano quindi sul server le classi logiche del programma, a partire dal dominio. Ogni classe corrisponde ad un oggetto del dominio, presentando i valori e le relazioni dei componenti come attributi dell’oggetto.




Entity Framework Core di .Net(EFCore) è una libreria di C\# che permette di unire le classi logiche del programma alle tabelle del database. Fornisce un’astrazione logica del collegamento con il database e le richieste relative, fornendo una rappresentazione di alto livello delle connessioni sottostanti. 


Una volta collegato il server con il database tramite le stringhe di connessione salvate sull’Azure Key Vault, sono state definite le proprietà tra le varie entità, per poi inizializzare in automatico la struttura del database. Le modifiche alla struttura del database vengono infatti generate automaticamente da EFCore in seguito alla creazione o alla modifica degli attributi degli oggetti. Questo permette di star dietro agli aggiornamenti, generando e salvando le modifiche da applicare ad ogni modifica delle proprietà del dominio.

Per la riduzione del carico computazionale richiesto da elementi con tante relazioni si utilizza la tecnica del lazy loading. La tecnica del Lazy Loading consiste nel richiedere i dati delle relazioni di un elemento solo quando strettamente necessario. La sua realizzazione tramite EFCore è attuata grazie alla proprietà virtual, che permette di gestire un oggetto con un riferimento al database richiedendo i dati delle sue relazioni solo quando viene espressamente richiesto.


\clearpage
\subsection{ Cache Locale}

Richiedere e ottenere dati dal server comporta ritardi che impattano sulle prestazioni. 
Se la tecnologia del dispositivo lo permette, è utile salvare copie delle informazioni usate più frequentemente nella memoria locale del dispositivo. Nel momento di una interazione che richiede la ricerca dei dati, si possono fornire le informazioni disponibili in copia, per poi eventualmente aggiornarle se si rivelassero imprecise. Questo permette un tempo di risposta apparente minore e una migliore esperienza utente.
\subsubsection{ Creazione della memoria locale}
Non tutti i dispositivi permettono una gestione della memoria a lungo termine. In particolare, i browser web non presentano questa funzionalità, ma i dispositivi mobili e i programmi si. 

Per motivi di prestazioni, il framework Flutter nativamente mantiene lo stato di un componente solo il tempo strettamente necessario per la sua esecuzione. Il tempo di vita dello stato coincide normalmente con quello del componente a cui è associato. La normale persistenza degli elementi e dei dati ha durata limitata. Si rende necessaria la creazione e la gestione di una cache locale che permetta di mantenere i dati ricevuti dal server anche in seguito al termine di servizio dei componenti.

\begin{figure}[h!]
    \centering
    \includegraphics[width=\textwidth]{FrontProviderClassDiagram.png}
    \caption{Classi provider all'interno dell'applicazione}
\end{figure}	
La cache locale viene suddivisa in base alle classi logiche del dominio, ed è accessibile tramite servizi dedicati. In particolare, grazie a componenti di tipologia provider è possibile aggiornare in automatico le parti dell’applicazione interessate dalle modifiche. Al termine di una richiesta dati al server, il provider viene notificato, salvando il dato in memoria e scatenando un aggiornamento a catena sui componenti interessati.

La cache deve essere unica e disponibile in tutto il programma. 
I provider attraverso cui si realizza la cache locale seguiranno il pattern singleton, che garantisce l’unicità dell’elemento all’interno del programma. Una volta creati all’avvio del programma, infatti, tutti gli elementi dell’applicazione avranno accesso agli stessi dati, quando necessario.

\begin{figure}[h!]
    \centering
    \includegraphics[width=\textwidth]{Providers.png}
    \caption{Esempio di interazione logica tra i componenti del client}
\end{figure}	

eventualmente salvataggio delle modifiche offline per caricarle quando si torna online. 

\subsubsection{ Allineamento con la memoria centrale}

il client ha la responsabilità di tenere allineata la propria cache ai valori della memoria centrale. Per quanto possa fare temporaneamente affidamento sulle risorse che ha salvato in cache per ridurre i tempi di risposta con l’utente, la loro validità dipende dalla certezza che corrispondano ai dati ufficiali. 

Ogni elemento avrà associato l’ultimo momento in cui è stato aggiornato. Allo stesso modo, viene salvato l’ultimo momento in cui è stata attivamente effettuata una richiesta esplicita di aggiornamento.  
Ad ogni avvio dell’applicazione il client invierà una richiesta al server per ricevere tutti gli elementi che hanno subito modifiche successivamente al momento dell’ultimo aggiornamento. 

La necessità di avere aggiornati tutti gli elementi della cache nel minor tempo possibile dipende anche dalla tipologia dell’elemento. Bisogna identificare gli elementi per il quale l’allineamento è critico per il corretto funzionamento dell’applicazione, come invece quelli che svolgono ruoli secondari e possono essere aggiornati anche in un secondo momento. Ad esempio, la modifica della durata di un evento è necessario venga rilevata il prima possibile, mentre la modifica della foto associata ad un profilo ha vincoli di aggiornamento in memoria locale molto più rilassati.

Nel caso di dati critici si implementa un meccanismo che esegue, periodicamente, una richiesta degli elementi modificati nell’arco di tempo dall’ultimo aggiornamento, per poi aggiornare il loro valore. Invece, nel caso di elementi secondari, si possono richiedere allineamenti direttamente nel momento in cui vengono posti espressamente in attenzione dato l’utilizzo dell’app.


\subsection{Aggiornamenti}

Data la natura condivisa dell’applicazione risulta fondamentale che gli utenti siano aggiornati in tempo reale sulle modifiche applicate agli eventi, oltre ad essere una prerogativa di tutte le applicazioni moderne, anche per migliorare la user experience. lo spostamento di un appuntamento, la conferma di una presenza o la modifica del luogo di appuntamento sono elementi critici che è bene che gli utenti vengano informati il prima possibile. 

Il cloud fornisce strumenti per il supporto e lo sviluppo di queste funzionalità, ma, oltre a dover individuare la tecnologia più adatta, bisogna anche essere in grado di integrarla nel resto del progetto.

\subsubsection{ Scelta della tecnologia}


La comunicazione con il server finora implementata si basa sul protocollo Hypertext Transfer Protocol (HTTP). HTTP prevede un fornitore di servizi (il server) mettere a disposizione una porta ad un indirizzo fisso rimanendo in attesa di eventuali utilizzatori (client) che, interfacciandosi attivamente alla porta disponibile,  espongono le loro richieste.
La riduzione delle comunicazioni al minimo indispensabile, oltre a non richiedere al server alcuna conoscenza del client, rende il protocollo pratico e scalabile.

Questa dinamica però impedisce ai client di essere notificati di eventuali modifiche apportate, a meno di richieste periodiche frequenti che comportano un sovraccarico da entrambe le parti. Inoltre, l’inversione dei ruoli non è applicabile in quanto i client cambiano costantemente l’indirizzo a loro associato, così come è impossibile distinguere se il dispositivo abbia terminato la connessione o se sia un guasto di altro tipo. 

Si necessita una comunicazione che mantenga in costante contatto i client con le modifiche del server, permettendo una trasmissione attiva degli aggiornamenti.
A basso livello, il protocollo che permette una comunicazione continua più adatto alle tecnologie comunemente diffuse è quello delle WebSocket. Tramite WebSocket infatti si crea un canale diretto tra le parti che consente una comunicazione istantanea. 

Alla necessità di supportare il protocollo delle WebSocket e di inviare istantaneamente i messaggi, si aggiunge la possibilità di creare molteplici canali specifici per indirizzare correttamente le comunicazioni ai soli interessati.


\begin{figure}[h!]
    \centering
    \includegraphics[width=\textwidth]{AzureMessagingServices.png}
    \caption{I servizi di comunicazione istantanea proprietari di Azure}
\end{figure}	


Per individuare la tecnologia più adatta ad aggiornare gli utenti, tra le tante che offrono servizi di collegamento istantaneo tra tecnologie, è fondamentale comprendere  gli scopi per cui sono nate e che problemi quindi risolvono. Infatti, ogni servizio è stato progettato per affrontare specifiche sfide che si differenziano sia per la natura dei servizi a cui si rivolgono che per  modalità di approccio .

La natura degli attori per cui il servizio si specializza determina le prestazioni di scalabilità e le integrazioni supportate. Bisogna quindi considerare la località e la natura delle risorse: in-premise o sul cloud, se appartengono alla stessa piattaforma o se devono comunicare internamente.
Ad esempio, un servizio pensato per collegare tantissimi dispositivi distribuiti con limitato potere computazionale, come nel caso dell’Internet of Things, fornirà  supporto a connessioni esterne e a protocolli standard, e prevederà un’elevata quantità di richieste di limitate dimensioni e frequenza. Viceversa, la necessità di creare una comunicazione tra un numero ristretto di server con prestazioni elevate comporta la creazione di flussi di dati importanti, magari gestiti internamente all’ambiente cloud, astraendo la tecnologia necessaria.

I servizi si differenziano però anche per le caratteristiche delle connessioni gestite.
Proprietà fondamentale è la natura delle comunicazioni. I canali possono essere infatti unidirezionali, permettere la comunicazione da entrambe le parti o implementati come flussi di eventi, in cui le comunicazioni possono essere inviate e ricevute da molteplici attori, senza che il destinatario sia noto al mittente. Inoltre, alcuni servizi offrono la possibilità di individuare categorie di clienti specifiche a cui eventualmente inviare notifiche mirate. Infine, bisogna prendere in considerazione la necessità di persistenza delle comunicazioni, che fornisce, oltre all’aggiornamento in tempo reale, anche la possibilità di recuperare modifiche passate.

Nel progetto si necessita di un servizio che supporti le WebSocket e che permetta di creare una molteplicità di canali unidirezionali differenti. In particolare, deve essere il più indipendente possibile dagli attori con cui comunica per poter garantire il maggior supporto possibile. Dovendo coprire solo le notifiche di aggiornamento, senza responsabilità di rintracciabilità dei dati, la presenza della persistenza non è necessaria.

Gratuito per le prime 20 connessioni, ma eventualmente scalabile per soddisfare ulteriori carichi, il servizio individuato per la gestione delle notifiche in tempo reale è Azure Web Pub Sub (AWPS). Permette infatti la creazione di canali tramite WebSockets e l’integrazione con le Azure Function. Supporta la creazione di canali, sia unidirezionali che bidirezionali, su cui pubblicare eventi, a cui gli utenti possono collegarsi per ricevere gli aggiornamenti. Non prevede l’utilizzo di persistenza ma gestisce completamente la scalabilità e l’affidabilità del sistema.



\subsubsection{Integrazione}

L’integrazione di Azure Web Pub Sub deve avvenire sia con il server che con i devices degli utenti. Seguendo il modello publish subscribe, ogni client si connetterà ad un canale in sola lettura, ricevendo tutti i dati che verranno pubblicati su di esso. Il server avrà il compito di interfacciarsi con il servizio per pubblicare i dati sui canali interessati.

La scelta della definizione del canale deriva da un’ulteriore analisi del dominio. 
Il soggetto interessato alle modifiche sottoposte a notifica è il profilo. 
Se però si creassero i canali in relazione ai profili ogni dispositivo (che riassume l’interazione di un utente) dovrebbe mantenere una connessione per ogni profilo collegato all’utente. La creazione di un canale per ogni device allo stesso modo risulta estremamente inefficiente, in quanto, oltre ad introdurre nuovi requisiti per garantire la tracciabilità dei dispositivi, ne richiederebbe di creazione e gestione in numero elevato. Per queste ragioni i canali verranno creati uno per utente, garantendo inoltre che gli unici utenti a ricevere le notifiche ne posseggano effettivamente l’accesso adeguato.

A seguito di una richiesta che comporta la notifica ai profili interessati, il server avrà il compito di interfacciarsi con AWPS per affidargli le comunicazioni relative. Tuttavia AWPS non supporta la capacità di unire gli elementi in base alle loro relazioni, per cui la responsabilità di trovare gli utenti interessati ricade sul server. 
Ad esempio, la modifica di un evento comporta la notifica a tutti i profili relativi, e quindi una comunicazione a tutti gli utenti che ne posseggono i permessi di notifica per la particolare azione su detti profili. 

L’operazione di ottenimento degli utenti coinvolti data l’azione svolta (nell’esempio, la modifica di un evento) verrà eseguita in un’altra Azure Function dedicata che si occuperà poi, per ogni utente coinvolto, di comunicare  al server AWPS il messaggio da notificare. L’eventuale fallimento della operazione viene inserito tra i log e risulterà durante i monitoraggi, senza coinvolgere la funzione principale.

		
\begin{figure}[h!]
    \centering
    \includegraphics[width=\textwidth]{IIModificaEvento2.png}
    \caption{Interazione delle Azure Functions con AWPS}
\end{figure}	

La ricezione della notifica sul dispositivo dell’utente può comportare la richiesta al server dell’elemento modificato. La scelta di recuperare i dati tramite il server invece di includere i dati direttamente all’interno della notifica permette di uniformare il formato delle notifiche, semplificando la loro gestione e velocizzando l'invio; di diminuisce il volume dei dati trasmessi tramite WebSocket, garantendo la scalabilità delle notifiche; di migliorare la gestione degli errori da parte del server rendendoli più affidabili e di garantisce la possibilità di recuperare solo l’ultimo aggiornamento, riducendo il consumo totale dei dati.


\begin{figure}[h!]
    \centering
    \includegraphics[width=\textwidth]{IIAggiornaEvento.png}
    \caption{Interazione tra AWPS e un client}
\end{figure}	

\clearpage