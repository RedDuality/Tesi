\chapter*{Ringraziamenti}
\addcontentsline{toc}{chapter}{Ringraziamenti}

Scrivere non è il mio forte, come penso questo documento attesti.\\
Vi chiedo dunque di accettare queste umili parole
come scarso tentativo di riconoscere e apprezzare quanto devo
alle persone che mi sono state accanto,
i cui sentimenti sono però più alti di quanto io
abbia la capacità di trasmettere su un foglio di carta.\\
\\
Ritengo che il primo ringraziamento sia corretto venga rivolto a mio babbo.
Questo percorso sarebbe stato molto diverso ed
estremamente più faticoso senza il suo silenzioso ma costante supporto.
La tranquillità e l'assenza di ulteriori preoccupazioni
che mi hanno accompagnato durante tutta l'università sono
i fattori che hanno più contribuito al raggiungimento di questo traguardo.
E sono merito suo.\\
\\
Sapere che c'è qualcuno su cui puoi contare,
che ci sarà sempre ad aspettarti, 
è una cose più belle che non riesco ancora ad apprezzare abbastanza.
Grazie quindi a Francesco, a Nicolò e a Matteo, che purtroppo ha deciso di tralocare in Giappone.
Grazie a Guglielmo, fida spalla su cui discutere delle cose più disparate,
nascondendo l'interesse e la noia dietro la stessa faccia.
Grazie a Pietro, compagno di studi e di nottate,
stratega e cecchino.
Grazie a Camilla, che mi ha introdotto a questa splendida compagnia.\\
\\
Durante il corso, dentro o fuori le aulee universitarie,
ho conosciuto e incrociato un sacco di persone,
che mi hanno colpito e modellato con i loro sorrisi,
ragionamenti o obiezioni.
Dato il fatto che magari non ci siamo mai presentati,
o, molto più probabilmente, non ricordo i loro nomi o
ancora me li sono persi durante la stesura di questa lista,
ringrazio tutte le persone che hanno reso questa esperienza unica, 
ma non si ritrovano citate.
\clearpage
Grazie ai compagni di bevute, di giochi mentali o di semplici giochi.
Grazie delle serate spensierate, dei complimenti 
e del supporto che abbiamo saputo dimostrare in gruppo.
Sto parlando di voi, Fjona, Omar, Leandro, Bruno, Eleonora e Giuseppe!(Omissis...)\\
\\
Con un sostegno e una sincera amicizia
sono stato accompagnato già dagli anni della triennale
da un gruppo di ragazze uniche,
con cui condividere sudore ma anche risate,
sorrisi e piroette finite a terra.
Grazie a Marina, Paola, Roberta, Marinella,
Francesca, Gaia, Anita, Eylul e a tutto il gruppo di danza.\\
\\
Ci sono amicizie che sono diventate stabili e durature,
che non importa come siano nate, ma dove mi abbiano portato.
Un grazie speciale a Elisa, Jacqueline, Andrea, Ada, Lorenzo, Patrizia, Andrea e Mattia.\\
\\
Un pensiero va a tutte le persone con cui ho legato negli ultimi tempi,
la cui amicizia si sta formando o desidera essere creata o mantenuta.\\
In particolare, 
a Emma Sofia, Natalia, Karina, Greta, Simona, 
Giovanni, Marta, Margherita, Matilde, Mirea, Laura e Nina;
grazie dell'interesse e dell'affetto con cui mi state accompagnando!\\
\\ 
Una compagnia che merita di essere nominata sono i gorilla dell'università.
Studenti speciali, che mi hanno accompagnato nelle aule e
con cui ho condiviso momenti di ansia ma anche di spensieratezza,
a qui purtroppo ho partecipato troppo poco frequentemente.
Grazie dell'esempio e del calore che avete saputo dimostrare.
So che sarò in buone mani se dovessi incrociarvi di nuovo sul mio percorso professionale.\\
\\
L'esperienza più memorabile di tutta l'università è stato lo studio in Erasmus.
Grazie a Vilnius, per avermi fatto provare cosa vuol dire non vedere il sole per mesi,
ma anche per il calore che è riuscita a farmi provare.
Calore che avrebbe avuto temperatura molto diversa se non avessi incontrato
Mujde, Lina, Laurine, Ane, Kathi, Manish, Sophie, Okan, Marco, Amane, Nour, Sinead e Marie-Sophie.
Ma anche con Lia, Maripili, Ardian, Joshua, Abbas, Karl, Alonso, Jakub, Sebastian e Simon.
\clearpage
Un supporto nascosto ma sempre presente 
risiede nei cuigini, nelle zie e nelle nonne.
Grazie a nonna Lucia e a nonna Luciana,
alla zia Ilaria, Serena e Angela.
A Giovanni, Biancamaria, Elisabetta, Riccardo e Marcolino.\\
\\
L'ultima parte è quella che si ricorda di più,
ed è qui che devo ringraziare la mia famiglia.
Grazie a Maria Giulia, 
che ha saputo starmi accanto come supporto concreto 
nei momenti più duri di questa esperienza,
mettendomi al primo posto nonostante tutte le preoccupazioni che stava vivendo.
Grazie a Francesco,
ci sono ricercatori che stanno ancora studiando
da dove arrivi tutta quella capacità di sopportarmi.\\
Grazie alla mamma,
la cui sua costanza, forza e presenza
sono state la metà di quelle impiegate per arrivare a questo obiettivo.\\
\\
Avrei voluto affiancare a queste parole foto che ci accomunassero,
ma purtroppo non sono uno che fa foto.
Spero venga implementato presto un'applicazione per risolvere questo problema.
   
